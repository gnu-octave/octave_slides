\begin{frame}{Wrapping up:}
\begin{itemize}
\itemsep2em
\item
What is GNU Octave?
\begin{itemize}
\itemsep1em
\item
High-level programming language, CLI/GUI Software and community.

\item
A \textbf{convenient interactive interface}
for many well-known and well-performing
numerical, graphical and utility libraries
written in \textbf{C/C++, Fortran, Python, Java, ...}

\item
\textbf{Free} to run, copy, distribute, study, change and improve.
\end{itemize}

\item
What it is \textbf{NOT}?
\begin{itemize}
\itemsep1em
\item
Not a \textbf{one-size-fits-all} solution for numerical computations.

\item
Not a compiled language, no transcompiler.

$\rightarrow$ There is an \textbf{interpreter overhead}.
\end{itemize}
\end{itemize}
\end{frame}



\begin{frame}
\begin{center}
\textbf{\Large Thank you for your attention!}\bigskip

\includegraphics[width=0.6\textwidth]{res/libreoffice/octave_community}

\textbf{\Large Questions?}
\end{center}
\vfill\footnotesize
Slides and sources available at:
{\color{DarkBlue}\url{https://github.com/octave-de/octave_slides}}
\end{frame}
