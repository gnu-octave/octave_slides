\begin{frame}{What is GNU Octave? - A community perspective.}
\begin{center}
\includegraphics[width=0.8\textwidth]{res/libreoffice/octave_community}
\end{center}
\end{frame}



\begin{frame}{What is GNU Octave? - A technical perspective.}
\vspace*{-1em}
\begin{center}
\includegraphics[width=0.9\textwidth]{res/libreoffice/octave_structure}
\end{center}
\vspace*{-1em}
\begin{exampleblock}{}
  $\quad$\small
  "[...]
  If I have seen further it is by standing on the sho[u]lders of Giants.
  [...]"
  
  $\quad${\small--- Isaac Newton (1676)}
\end{exampleblock}
\end{frame}



\begin{frame}{Why GNU \textbf{"Octave"}?}

\begin{itemize}
\itemsep2em
\item
About \textbf{1992} \textbf{\color{DarkBlue}John W. Eaton (jwe)}
starts development\\[0.5em]
\begin{itemize}
\itemsep1em
\item
since then in total about \textbf{440 contributors}
\item
currently about \textbf{9-14 active developers}\footnote{Contributions at Octave core and Forge within last month.}
\end{itemize}

\item
Named after \textbf{\color{DarkBlue}Octave Levenspiel} (1926-2017)\\[0.5em]
\begin{itemize}
\itemsep1em
\item
former professor of jwe
\item
famous for quick back-of-the-envelope calculations
\end{itemize}
\end{itemize}
\end{frame}


\begin{frame}{Why \textbf{"GNU"} Octave?}

\begin{itemize}
\itemsep1.5em
\item
"GNU" Octave since \textbf{1997} (version 2.0.6)

\item
\textbf{Vision} of the
\textit{GNU project}\footnote{Recursive: "GNU's Not Unix!"}
of the \textit{Free Software Fdn. (FSF)}:\\[1em]

\begin{quotation}
"[...] the users have the freedom to run, copy, distribute, study, change
and improve the software. [...] 'free' as in 'free speech,'
not as in 'free beer'. [...]"
\footnote{\url{https://www.gnu.org/philosophy/free-sw.html}}
\end{quotation}

\item
Using \textbf{infrastructure} (e.g. code hosting and bug tracking)
\begin{itemize}
\item
SourceForge (1999), GitHub (2008), ...
\end{itemize}

\item
\textbf{Sponsorship} "Working Together for Free Software Fund"
\end{itemize}
\end{frame}
