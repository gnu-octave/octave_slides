\documentclass{scrartcl}

\usepackage[utf8]{inputenc}
\usepackage[T1]{fontenc}
\usepackage{lmodern}
\usepackage{fancyhdr}
\usepackage{hyperref}
\usepackage[bottom=20mm]{geometry}

\fancypagestyle{plain}{}
\chead{JSIAM 11th Three-Group Meeting "Applied Mathematical Seminar"\\
University of Tokyo}
\cfoot{}

\title{Your own GNU Octave under the Xmas tree}
\subtitle{Finish your numerical experiments before New Year's Eve}
\author{OHLHUS, Kai Torben\footnote{email: \texttt{ohlhus@lab.twcu.ac.jp},
Graduate School of Science,
Tokyo Woman's Christian University}}
\date{December 24, 2019}


\begin{document}

\maketitle

\section*{About GNU Octave}

GNU Octave is a numerical computing environment
and a high-level programming language.
It provides a convenient interface for solving linear and nonlinear problems,
and for performing other numerical experiments.
The language is mostly compatible with MATLAB.
Besides 2D and 3D plotting capabilities,
Octave provides an integrated development environment for editing,
debugging, and profiling numerical algorithms.
Octave can be freely obtained, modified, and redistributed
under the terms of the GPL, the free and open source software license.

\section*{Outline}

This talk aims to provide a deeper knowledge in developing
fast numerical algorithms using Octave by investigating several possibilities
and limitations of that software.
Some topics covered are how to write fast Octave code by considering
vectorization, appropriate data types, and memory management strategies.
Concepts for organizing larger software projects
by using object-oriented programming and packages are presented
and how to conveniently interface existing C/C++ and Fortran code libraries.
This knowledge is a good start for the development of GNU Octave itself
and how to build custom versions,
which will be introduced in the remainder of the talk.

\section*{Preparation and material}

During the talk there will be small "hands on" sessions using Jupyter notebooks.
For the best experience of these sessions,
please bring your own laptop with Internet connection (e.g. eduroam).
The material can be obtained online\footnote{\url{https://github.com/octave-de/octave_slides/releases/tag/2019-12-24}}.
Please read the document \texttt{setup.pdf}.

For a limited number of about 15 participants,
we can provide online access to the TWCU JupyterLab server\footnote{\url{http://jupyter.math.twcu.ac.jp:8000}}.
In this case only an Internet connection is required.
The login data is provided at the day of the event
or can be obtained by the author beforehand.

\end{document}
